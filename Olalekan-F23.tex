% \documentclass{article}
% \usepackage{graphicx} % Required for inserting images
% \usepackage{url}

% \title{Git Assignment}
% \author{Zainab Olalekan}
% \date{September 2023}

% \begin{document}

% \maketitle

\section{Introduction}  \label{sec-introduction}


\begin{figure}[h]
    \begin{center}
        \centering
    \includegraphics[width=3in]{Olalekan.jpg}
    \caption{Personal Photo}
    \centering
    \label{fig:figure1}
    \end{center}    
\end{figure}

I am Zainab Olalekan, a second-year Ph.D. student working in the Secure Software Systems and Embedded System Security lab with Dr. Yanyan Zhuang and Dr. Gedare Bloom. Before this, I had limited exposure to research and mainly used available tools. I am taking the CS 6000 class to enhance my research skills and learn how to conduct impactful research while improving my critical thinking, reading, and writing abilities. I am currently investigating peripheral management in emulated embedded devices for better dynamic analysis. However, an attempt to write a survey paper on the topic proved ineffective. This is because of my inability to pass messages across and understand scientific articles properly. This course will equip me with the fundamental skills to conduct rigorous research and harness my potential in all facets of life because such skills are transferable. I have always wanted to try new things like hiking and traveling, but I hesitate. Over the summer, I tried getting out of my comfort zone by traveling to different states alone. It has given me the courage to try new things and keep traveling when the opportunity presents itself. By the end of this semester, I will have acquired the skills to conduct quality research to help me overcome some roadblocks I face and contribute to my field of interest.

\section{Research Repository}  \label{sec-research-repository}

\url{https://github.com/fuzzware-fuzzer/fuzzware.git}

The link above is an artifact for  Fuzzware~\cite{scharnowski2022fuzzware}. The project aims to manage firmware interaction with peripherals via fuzzing. Downloading and installing the code was straightforward because the repository included instructions.


\bibliographystyle{plain}

\bibliography{Olalekan-ref.bib}

\section{Questions}
\subsection{Question 1}
\paragraph{Hi Zainab, Would you like my research search analytics script? It could help identify papers. It only takes one really interesting paper and the references of that paper to start a research trend and relevant related works.}
\subsection{Question 2}
\paragraph{}

%\end{document}
