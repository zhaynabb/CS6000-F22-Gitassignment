\documentclass[a4paper]{article}

%% Language and font encodings
\usepackage[english]{babel}
\usepackage[T1]{fontenc}

%% Sets page size and margins
\usepackage[a4paper,top=3cm,bottom=2cm,left=3cm,right=3cm,marginparwidth=1.75cm]{geometry}

%% Useful packages
\usepackage{amsmath}
\usepackage{graphicx}
\usepackage[colorinlistoftodos]{todonotes}
\usepackage[colorlinks=true, allcolors=blue]{hyperref}

\title{Week 1 Journal}
\author{Kevin Cardenas}

%\begin{document}
\maketitle

\section{Introduction and Goals for CS6000}

My name is Kevin Cardenas and I am a Captain in the US Air Force. I have the unique opportunity to be a full-time PhD student for my job as an active duty military member. I taught in the Department of Computer and Cyber Sciences at the USAF Academy for three years and will be able to go back to teaching after completing my PhD. The most challenging aspect of my PhD will be the fact that the Air Force only grants me three years to complete my degree. I chose to get a PhD in Computer Science so I could become a better asset for the USAF Academy and broaden my experience in the realm of Data Science. This will most likely add to the challenge of completing this degree in three years, but I like challenges. 

One of my primary goals for this course is to learn how to quickly sift through large volumes of articles/conference papers/journals/etc to find pertinent information relating to my area of research. A secondary goal for this course is to have a well defined research topic by the end of the semester. These two goals should help me overcome the shortened time limitation the Air Force requires to complete a PhD.

\section{Git Repo}
    \href{https://github.com/serengil/deepface}{Deepface} is a lightweight face recognition and facial attribute analysis (age, gender, emotion and race) framework for python. It is a hybrid face recognition framework wrapping state-of-the-art models: VGG-Face, Google FaceNet, OpenFace, Facebook DeepFace, DeepID, ArcFace, Dlib and SFace.

Experiments show that human beings have 97.53% accuracy on facial recognition tasks whereas those models already reached and passed that accuracy level.

\begin{figure}[!htb]
\centering
\includegraphics[width=0.4\textwidth]{figures/Kevin.jpg}
\caption{\label{fig:me}Wedding day mess dress (6 Sept 2019).}
\end{figure}

\section{Questions}
\textbf{Question 1- Katrina} What made you interested in data science? 

%\end{document}