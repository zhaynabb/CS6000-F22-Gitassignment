\hypersetup{colorlinks=true, urlcolor=blue, linkcolor=black}

% My macros
\newcommand{\myfullname}{Carlos Eugenio Lopes Pires Xavier Torres}
\newcommand{\myshortname}{Carlos E. Torres}

% Title
\section{CS 6000 - Week 5 - Git Assignment}

Carlos Eugenio Lopes Pires Xavier Torres



\section{About me}

I am pursuing my PhD in Computer Science since Fall 2021 at the University of Colorado, Colorado Springs. I have worked with Dr. Xu in one of his research projects related to cybersecurity, secure and privacy-preserving data sharing. I created a system on that research that I will present in November at the ISTSS annual meeting in Los Angeles, CA.
I am currently involved in a research project with my new advisor, Dr. Oluwadare, in the Bioinformatics field, researching 3D chromosome and genome structure modeling and visualization.

\begin{wrapfigure}[11]{r}{0.25\textwidth}
    \centering
    \includegraphics[width=0.25\textwidth]{Torres-F23-photo.jpg}
    \caption{\small My photo}
\end{wrapfigure}

I am currently working full-time as a Principal Software Engineer at Oracle. I work with data science, big data, and full stack web development. My past experience includes working as a senior software engineer, expert in mobile and web solutions development with 22 years experience. Experienced in fast-paced environments, team leadership, client-facing and customer relationship. Also experienced in the whole software development process, from concept, requirements, analysis, database modeling, programming, testing, until production and maintenance.

There are a few ways to contact me and get to know more about my experience, projects, and career. Just visit these links:
\begin{itemize}
    \item LinkedIn: \url{https://linkedin.com/in/cetorres}
    \item GitHub: \url{https://github.com/cetorres}
    \item Twitter: \url{https://twitter.com/cetorres}
    \item Website: \url{https://cetorres.com}
\end{itemize}

\section{A Git repository related to my research}

HiC-GNN is a generalizable model for 3D chromosome reconstruction using graph convolutional neural networks. This work was done in my current research lab (Oluwadare Lab) by Van Hovenga and Oluwatosin Oluwadare. The repository (\url{https://github.com/OluwadareLab/HiC-GNN}) has the code referenced in their paper that is used to train the model and generate the .pdb file containing the predicted 3D structure corresponding to the input file. The input file must be a Hi-C contact map in either matrix format or coordinate list format.

% ---------------------------------
% INSTRUCTIONS FOR QUESTIONS
% Please make your questions below.
% Replace [Question 1] and [Question 2] with your actual questions.
% One question per person.
% ---------------------------------

\section{Questions}

\subsection{Nazmus: Can you let me know what are the algorithms you are using?}
Answer: On this Git repo I showed above, the algorithm used is a graph convolutional neural network with specific tuned parameters, loss function, and optimizer to train a model to read Hi-C input files, generalize, and generate .pdb file containing the predicted 3D structure. But in the lab, we use a variety of algorithms, either related to machine learning and deep learning like this, or others like particle swarm and maximum likelihood.

\subsection{[Question 2]}


Zainab: How are you finding the transition from security-related research to Bioinformatic? 

% \end{document}

