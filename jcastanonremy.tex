\documentclass[article]{IEEEtran}
\usepackage[utf8]{inputenc}
\usepackage{graphicx}
\usepackage{cite}
\usepackage{url}

\title{Week 4. GitHub Assignment \\
\large UCCS CS 6000 Computer Science Research}
\author{Jose Luis Castanon Remy}
\date{September 2022}

\begin{document}

\maketitle









\section{Myself}

During my experience working as a cybersecurity analyst, I have always faced different issue related to the field. I always thought no one had the interest to solve or even address this issues. Common issues that I have found while working are disinformation, disinterest and uncontrol. This constitutes my second goal. I would love to research and solve issues within the cybersecurity field that no one is paying attention to. One of this issues is the fact that many companies that I had the opportunity to work for, all lack a proper inventory of software leading to security issues.



In addition to my goals and expectations, I consider myself to be very creative. I love to paint. I consider myself an artist in multiple disciplines. I love charcoals and pencils. I am currently learning color theory as well as how to paint figurines and build dioramas. Apart from my artistic face, I love biking. It is the best way for me to add cardio to my gym routine without getting bored.

I will be completing my PhD in Cybersecurity.


\begin{figure}[htp]
    \centering
    \includegraphics[width=4cm]{UCCSProfileIMG.jpg}
    \caption{An image of myself}
    \label{fig:myself}
\end{figure}


\section{Git Code}
The research that I am currently working on is related to security and asoftware maintenance. It is a broad topic but I will be trying to narrow it down as I advance with my story.

Maintenance is a large part of every security team. Maintenance represented more than 50\% of the tasks and duties within the last security team that I wroked with. Part of the maintenance process was realted to cleaning. There is a broaldy knowwn app related to cleaning called CCleaner. Clenaing helps solving issues directly linked to unused files within installation folders, cache folders, etc. CCleaner helps detecting and cleaning up this files.

Currently, the tasks that CCleaner is able to perform is large and complicated. For this reason I have chosen the following Git repo, Winapp2, \url{https://github.com/MoscaDotTo/Winapp2}.

Winapp2 extends the cleaning routines of CCleaner. Winapp2 compiles a list of keys, as a conofiguration file, for CCleaner to check more applications to clean. This is hihgly related to my research as I would like to build somethig that is able to map the stack of software within a computer system. Winapp2 is not also comaptible with CCleaner but with other cleaning apps like BleachBit, System Ninja, Avira System Speedup, Tron and R-Wipe & Clean.


\section{Questions}
Please, leave your questions after this line.



\bibliographystyle{IEEEtran}
\bibliography{refs}

\end{document}
